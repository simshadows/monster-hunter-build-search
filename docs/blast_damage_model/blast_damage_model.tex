\documentclass{article}

\usepackage[margin=2.5cm, bottom=3.0cm]{geometry}
\usepackage[
    type={CC},
    modifier={by-sa},
    version={4.0},
    hyphenation={RaggedRight},
    ]{doclicense}

\usepackage{../_latex_includes/sharedpkg}

\def\TITLE{MHWI Blast Damage Model}

\setcounter{tocdepth}{3}
\setcounter{secnumdepth}{3}

\begin{document}

\thispagestyle{plain}
\MakeCustomTitle
\bigskip



\begin{center}%
\begin{minipage}[c]{12cm}%
    \doclicenseThis%
\end{minipage}%
\end{center}%
\bigskip


\section{Introduction}%
\label{sec:introduction}

The mathematical description of how status effects such as blast is well-described in \textit{How Status Works: Definitive Status Guide MHW/Iceborne} (\href{link}{https://www.youtube.com/watch?v=iIPfkvvbGwY}). However, I will attempt to summarize it here, in addition to clarifying some features of this game mechanic not mentioned in the video.

Status effects can be applied to monsters a variety of ways, such as attacking while using a weapon with a poison/paralysis/sleep/blast stat.

When a status is applied to a monster, that status is ``built-up". Only when this internal status application counter reaches certain thresholds does the status effect trigger (or \textit{proc}).

\subsection{Status Application}%
\label{sub:blast_application}

All monsters have unique tables regarding status effect application. Buildup usually involves four particular values:

\begin{itemize}
    \item \textit{Base Tolerance} (or simply \textit{base}): The threshold before the first proc.
    \item \textit{Tolerance Buildup} (or simply \textit{buildup}): Every time we hit the threshold, we increase the new threshold by this value.
    \item \textit{Tolerance Cap} (or simply \textit{cap}): The maximum threshold.
    \item \textit{Decay}: The internal status application counter decays at a constant rate over time.
\end{itemize}

To illustrate how these values work, let's consider Dodogama's sleep values:

\begin{itemize}
    \item Base:    $150$
    \item Buildup: $100$
    \item Cap:     $550$
    \item Decay:   $-\frac{5}{10} \text{ per second}$
\end{itemize}

Additionally, let's specifically consider the following quest values:

\begin{itemize}
    \item Base Multiplier: $1.25$
    \item Buildup/Cap Multiplier: $1.90$
\end{itemize}

Thus, our effective sleep values for that particular quest are:

\begin{itemize}
    \item Base:    $150 \times 1.25 = 187.5$
    \item Buildup: $100 \times 1.90 = 190$
    \item Cap:     $550 \times 1.90 = 1045$
    \item Decay:   $-\frac{5}{10} \text{ per second}$
\end{itemize}

Our first threshold is $188$, so we must apply $188$ sleep in order to get the first sleep proc.

After the first sleep proc, our new threshold is $188+190=378$, so we must apply $378$ sleep for the second proc.

After the second sleep proc, our new threshold is $378+190=568$, so we must apply $568$ sleep for the third proc.

This follows an arithmetic progression up until the cap, where it is clipped:
\begin{equation*}
    188
    \longrightarrow 378
    \longrightarrow 568
    \longrightarrow 758
    \longrightarrow 948
    \longrightarrow \cancel{1138} \; 1045
    \longrightarrow 1045
    \longrightarrow 1045
    \longrightarrow \cdots
\end{equation*}

\textit{IMPORTANT: This is a draft. I actually don't know how the rounding works, or other variables that status decay depends on, but my blast model is a continuous model, and blast doesn't decay, so it doesn't affect the remainder of this document.}

\subsection{Blast Buildup}%
\label{sub:blast_buildup}

Blast is a status effect that \textit{does not decay} (i.e. decay is 0 blast per second).

Other than that, blast base/buildup/cap values work as expected.

\subsection{Blast Procs}%
\label{sub:blast_procs}

A blast proc deals fixed damage, regardless of monster (100 in Low Rank, 120 in High Rank, and 300 in Master Rank). The damage is dealt instantly, and the status can immediately continue being built-up.


\section{The Continuous Blast Damage Model}%
\label{sec:the_continuous_blast_damage_model}


\subsection{Input Variables}%
\label{sub:input_variables}

Our model is based on the idea that a single specified attack always deals a predictable amount of average damage and blast application each time the attack is performed.

For example, if a greatsword draw attack dealt 100 raw damage and 5 status buildup on average, we can expect that using the attack $n$-times would deal $n \times 100$ raw damage and $n \times 5$ status buildup on average.

These variables shall be:
\begin{itemize}
    \item $\rho$ (average raw damage of an attack)
    \item $\sigma$ (average status application of an attack)
\end{itemize}

Conceptually, we can try counting how many of the specified attack (such as the greatsword draw attack) we have to deal before we slay the monster, so we consider monster health to be another variable:
\begin{itemize}
    \item $H$ (monster total health)
\end{itemize}

If we ignore blast damage and assume all our damage comes from raw, it is trivial to see that we'd require $\frac{H}{\rho}$ uses of the specified attack in order to kill the monster. However, it gets tricky when we factor in blast.

The variables related to blast are:
\begin{itemize}
    \item $a_1$ (blast base)
    \item $d$ (blast buildup)
    \item $c$ (blast cap)
    \item $P$ (blast damage \textit{per proc})
\end{itemize}

Our model will attempt to find the average blast damage per attack.


\subsection{Pre-Cap Model Derivation}%
\label{sub:pre_cap_model_derivation}

The following two equations are given for arithmetic progressions:
\begin{gather}
    a_n = a_1 + \parens*{n - 1} d \\
    S_n = \frac{n}{2} \parens*{a_1 + a_n}
\end{gather}
where $a_1$ is the first element of the sequence, $a_n$ is the $\Nth{n}{th}$ element, $d = a_{i+1} - a_{i}$, and $S_n$ is the sum of the first $n$ elements.

Equivalently, $a_n$ is the threshold for the $\Nth{n}{th}$ blast proc, and $S_n$ is the total blast that must be applied to reach the $\Nth{n}{th}$ blast proc.

Eliminating $a_n$:
\begin{align}
    S_n &= \frac{n}{2} \parens*{a_1 + a_1 + \parens*{n - 1} d} \\
        &= \frac{n}{2} \parens*{2 a_1 + \parens*{n - 1} d}
\end{align}

Solving for number of procs $n$:
\begin{align}
    2 S_n &= n \parens*{2 a_1 + \parens*{n - 1} d} \\
          &= 2 n a_1 + n^2 d - n d
\end{align}
\begin{align}
    0 &= 2 n a_1 + n^2 d - n d - 2 S_n \\
      &= n^2 d + n \parens*{2 a_1 - d} - 2 S_n
\end{align}

Since $n \ge 0$:
\begin{align}
    n &= \frac{- \parens*{2 a_1 - d} + \sqrt{\parens*{2 a_1 - d}^2 - 4 d \times \parens*{- 2 S_n}}}{2 d} \\
      &= \frac{d - 2 a_1 + \sqrt{\parens*{2 a_1 - d}^2 + 8 d S_n}}{2 d}
\end{align}

Let $R_n$ and $B_n$ denote the total raw and blast damage (respectively), given $n$ blast procs.

Since $P$ is blast damage per proc, total blast damage is simply:
\begin{equation}
    B_n = Pn
\end{equation}

Thus, we can calculate $B$ as:
\begin{equation}
    B_n = P \frac{d - 2 a_1 + \sqrt{\parens*{2 a_1 - d}^2 + 8 d S_n}}{2 d}
\end{equation}

Since we know that we apply $\sigma$ status for every $\rho$ raw damage, it is trivial to calculate $R_n$ using:
\begin{equation}
    \frac{\rho}{\sigma} = \frac{R_n}{S_n}
\end{equation}

Thus, we can introduce $R_n$ into our equation. Additionally, we no longer care about counting blast procs, so we can simplify the notation of total raw and blast damage to $R$ and $B$:
\begin{equation}
    B = P \frac{d - 2 a_1 + \sqrt{\parens*{2 a_1 - d}^2 + 8 R d \frac{\sigma}{\rho} }}{2 d}
\end{equation}

In order to slay the monster, total raw and blast damage must add to equal the total health pool $H$:
\begin{equation}
    H = R + B
\end{equation}

Substituting into $R$:
\begin{gather}
    B = P \frac{d - 2 a_1 + \sqrt{\parens*{2 a_1 - d}^2 + 8 \parens*{H - B} d \frac{\sigma}{\rho} }}{2 d}
    \\
    \frac{2Bd}{P} = d - 2 a_1 + \sqrt{\parens*{2 a_1 - d}^2 + 8 \parens*{H - B} d \frac{\sigma}{\rho} }
    \\
    \frac{2Bd}{P} + 2 a_1 - d = \sqrt{\parens*{2 a_1 - d}^2 + 8 \parens*{H - B} d \frac{\sigma}{\rho} }
    \\
    \parens*{\frac{2Bd}{P} + 2 a_1 - d}^2 = \parens*{2 a_1 - d}^2 + 8 \parens*{H - B} d \frac{\sigma}{\rho}
    \\
    B^2 \frac{4 d^2}{P^2} + 2 \frac{2Bd}{P} \parens*{2 a_1 - d} + \cancel{\parens*{2 a_1 - d}^2}
        = \cancel{\parens*{2 a_1 - d}^2} + 8 H d \frac{\sigma}{\rho} - 8 B d \frac{\sigma}{\rho}
    \\
    B^2 \frac{4 d^2}{P^2} + B \frac{4 d}{P} \parens*{2 a_1 - d}
        = 8 H d \frac{\sigma}{\rho} - 8 B d \frac{\sigma}{\rho}
    \\
    B^2 \frac{4 d^2}{P^2} + B \parens*{\frac{4 d}{P} \parens*{2 a_1 - d} + 8 d \frac{\sigma}{\rho}}
        = 8 H d \frac{\sigma}{\rho}
    \\
    B^2 \frac{4 d^2}{P^2} + B 4 d \parens*{\frac{2 a_1 - d}{P} + \frac{2 \sigma}{\rho}}
        = \frac{8 H \sigma d}{\rho}
    \\
    B^2 + B \frac{P^2}{d} \parens*{\frac{2 a_1 - d}{P} + \frac{2 \sigma}{\rho}}
        = \frac{2 H P^2 \sigma}{\rho d}
    \\
    B^2 + 2 B \frac{P^2}{2d} \parens*{\frac{2 a_1 - d}{P} + \frac{2 \sigma}{\rho}}
        = \frac{2 H P^2 \sigma}{\rho d}
    \\
    B^2 + 2 B \frac{P^2}{2d} \parens*{\frac{2 a_1 - d}{P} + \frac{2 \sigma}{\rho}}
        + \parens*{\frac{P^2}{2d} \parens*{\frac{2 a_1 - d}{P} + \frac{2 \sigma}{\rho}}}^2
        = \frac{2 H P^2 \sigma}{\rho d}
        + \parens*{\frac{P^2}{2d} \parens*{\frac{2 a_1 - d}{P} + \frac{2 \sigma}{\rho}}}^2
    \\
    \parens*{B + \frac{P^2}{2d} \parens*{\frac{2 a_1 - d}{P} + \frac{2 \sigma}{\rho}}}^2
        = \frac{2 H P^2 \sigma}{\rho d}
        + \parens*{\frac{P^2}{2d} \parens*{\frac{2 a_1 - d}{P} + \frac{2 \sigma}{\rho}}}^2
    \\
    B + \frac{P^2}{2d} \parens*{\frac{2 a_1 - d}{P} + \frac{2 \sigma}{\rho}}
        = P \sqrt{
            \frac{2 H \sigma}{\rho d}
            + \parens*{\frac{1}{2d} \parens*{\frac{2 a_1 - d}{P} + \frac{2 \sigma}{\rho}}}^2
        }
    \\
    B
        = P \sqrt{
            \frac{2 H \sigma}{\rho d}
            + \parens*{\frac{1}{2d} \parens*{\frac{2 a_1 - d}{P} + \frac{2 \sigma}{\rho}}}^2
        }
        - \frac{P^2}{2d} \parens*{\frac{2 a_1 - d}{P} + \frac{2 \sigma}{\rho}}
        \label{eq:total_blast}
\end{gather}

Thus, we now have an expression for total blast damage, given the monster's total health pool.

Now, we want to find out the average amount of blast damage dealt per attack, given the attack's raw damage.

Let $\beta$ denote the average blast damage per attack.

It is trivial to use the ratio between $R$ and $B$:
\begin{equation}
    \frac{\beta}{\rho}
    = \frac{B}{R}
    = \frac{B}{H - B} 
\end{equation}

Thus, we get an expression for $\beta$ in terms of $B$, $H$, and $\rho$:
\begin{equation}
    \beta = \rho \frac{B}{H - B}
\end{equation}

Since we can use \eqref{eq:total_blast} to calculate $B$ in terms of model input values, we now have an expression for calculating average blast damage per attack, given only our model input values.


\section{Model Extensions}%
\label{sec:model_extensions}

\subsection{Combos}%
\label{sub:combos}

It turns out that if you can predict the average amount of raw damage and status application of a single attack, you can also do it for a given sequence of attacks.

Thus, $\rho$ and $\sigma$ can instead be used as values for a given sequence attacks rather than just a single attack.


\subsection{Target Health}%
\label{sub:target_health}

$H$ doesn't actually have to be the monster's total health pool if we don't intend on slaying it. For example, we can instead make $H$ the damage required in order to capture the monster.

Multiplayer might also be considered here since multiple players share the full health pool of the monster. For example, if we assume four perfectly equal players, then we might want to divide the full 4-player health pool by 4.


\subsection{An Exact Blast Model}%
\label{sub:exact_blast_model}

\textit{(This section will be written later!)}


\end{document}
